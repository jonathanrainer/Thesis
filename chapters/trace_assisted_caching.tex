\chapter{Trace Assisted Caching}
\label{chap:trace-assisted-caching}

Having now considered the literature we move on to a theoretical explanation of the new technique this thesis proposes. We will start by exploring the motivation for this new technique; give a theoretical, processor agnostic, description of how to works; and follow that with a section that explains how this technique will bring benefit to hardware that implements this system.

\section{Motivation}

If we return to the central question of this thesis, we are trying to understand how memory latency can be reduced to reduce overall program runtime. However the results of the literature review in the previous chapter appear to show that we are blocked along all the avenues considered as a route to this goal. For example, cache policies definitely show some ability to reduce latency by making smarter decisions about which elements should be replaced and when that should happen. However there is an upper limit on their ability to reduce latency, namely the \texttt{OPT} ceiling, and much of this gap has been closed with the research we've seen. As a result more work in this area is going to follow the law of diminishing returns, leading a lot of researchers to abandon work on cache policies because they are 'good enough` \cite{podlipnigSurveyWebCache2003}. Moreover, \texttt{OPT} cannot be implemented because perfect clairvoyance is impossible in general and even if it could be the effectiveness of \texttt{OPT} is still not 100\%. So even if we could implement a version of \texttt{OPT} that worked there are serious questions as to whether we'd want to or whether there were more effective ways to reduce latency in other areas.

\subsection{Defining the Key Problem}

If we look at the other key areas of the literature we see the same story over and over again. Either the economics of the decision don't balance, i.e. we need a very large hardware budget to achieve a comparatively small drop in latency; or techniques don't have enough information available to them to make good decisions. So if we boil down these problems what is the recurring theme? 

% % % Rework this section so we talk about a lack of information and then a belief in synchronicity. Then talk about what information would actually be useful if we broke the assumption. Then where it could come from. Finally what harnessing it would look like

I believe the reason these techniques are not as performant as they might be is that they are making decisions based upon an incomplete model of the world as it exists which leads them to either make wrong decisions or to not react quickly enough to change. Caches for example are designed deliberately to be relatively simple. This gives the benefit that they take up very little silicon on a processor die but also that they will be fast due to a short critical path. Because of these two constraints any attempts to give them more information are difficult and so their decisions will always be based on the stimulus they are given (cache misses) rather than other more indicative behaviour. Pre-fetching too is very similar, because most prefetching units do not have a level of introspection into the program as it's executing they cannot utilise dynamic information to make better decisions. As a result they are left either use the past as an indicator of the future or to assume the future follows a pre-defined pattern. Neither of these is ideal and it's clear that with more information much better decisions could be made.

If we wanted to solve this problem and give more information to these techniques so they could make better decisions, what information would they need? The most useful thing that could be given to them 

\subsection{What Information Would Be Useful?}

If we side step this key problem slightly and move to think about tracing as a technique. What does tracing give us?

\subsection{Tracing: A Solution}

\subsection{A Vision of our Destination}

\section{A Schematic Design}

\subsection{Trace Recorder}

\subsection{Intelligent Cache \& Memory System}

\section{Justification of Success}